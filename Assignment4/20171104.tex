\documentclass[a4paper,12pt]{article}
\usepackage[english]{babel}
\usepackage[utf8]{inputenc}
\usepackage{fancyhdr}

\begin{document}
\title{\textbf{\underline{Richard Hamming}}}
\author{by Eesha Dutta}
\date{}
\maketitle
\begin{center}
\textsl{"What you learn from others you can use to follow.\\
What you learn for yourself you can use to lead."}
\end{center}
\rule{\textwidth}{0.4pt} 

\section{\underline{Introduction}}
Richard Wesley Hamming (February 11, 1915 to January 7, 1998) was an American mathematician whose work has many implications for computer engineering and telecommunications. He received the Turing award for his work on numerical methods, automatic coding systems, error-detecting and error-correcting codes.\\
He was born in Chicago, Illinois on February 11, 1915, the son of  Richard J. Hamming and Mabel G. Redfield and grew up there. He obtained a BSc degree from University of Chicago, majoring in Mathematics.\\
He obtained a MA degree from the University of Nebraska in 1939, and then entered the University of Illinois at Urbana-Champaign, where he wrote his doctoral thesis on \textit{Some Problems in the Boundary Value Theory of Linear Differential Equations.}\\
After earning his doctorate, Hamming married Wanda Little on September 5, 1942. He delivered his last lecture in December 1997, just a few weeks before he died from a heart attack on January 7, 1998.\\
\rule{\textwidth}{0.4pt}

\section{\underline{Work}}
Richard Hamming is best known for his work at Bell Labs on error-detecting and error-correcting codes. His fundamental paper on this topic, \textit{Error detecting and Error correcting codes,} created an entirely new field within information theory. \textbf{Hamming codes, Hamming distance} and \textbf{Hamming metric,} used today in coding theory and other areas of mathematics, all originated in this classic paper and are of ongoing practical use in computer design.
\subsection{Manhattan Project}
In 1945, during World War II, Hamming  joined the Manhattan Project, a U.S. government research project at Los Alamos and was in charge of the IBM calculating machines that played a vital role in the project.\\
After the Manhattan project ended, he remained at Los Alamos to understand the importance of what had been achieved and why it was so successful. 
\subsection{Bell laboratories}
In 1946 Hamming joined the mathematics department at the Bell Telephone Laboratories. Hamming spent much of his time with \textbf{calculating machines.} Before he went home on one Friday in 1947, he set the machines to perform a long and complex series of calculations over the weekend, only to find  that the calculation had errored off due to an error early in the process. Since digital machines manipulated information as sequences of zeroes and ones known as  "bits", even if a single bit in a sequence was wrong, then the whole sequence would be. To detect this, a parity bit was used. "If the computer can tell when an error has occurred," Hamming reasoned, "surely there is a way of telling where the error is so that the computer can correct the error itself."
\subsection{Other Work}
In 1956 Hamming worked on the IBM 650, an early vacuum tube, drum memory, computer. His work led to the development of the \textbf{L2} programming language, one of the earliest computer languages. He worked on numerical analysis, especially integration of differential equations. The Hamming spectral window, widely used in computation, is a special type of digital filter designed to pass certain frequencies and discriminate against closely related frequencies.\\
Hamming served as president of the Association for Computing Machinery from 1958 to 1960. He held visiting or adjunct professorships at Stanford University, City College of New York, University of California and Princeton University.\\
\rule{\textwidth}{0.4pt}

\section{\underline{Hamming Codes}}
Hamming codes are a family of linear error-correcting codes which can detect up to two-bit errors or correct one-bit errors without detection of uncorrected errors. Hamming codes are perfect codes, that is, they achieve the highest possible rate for codes with their block length and minimum distance of three.
 \subsection{Challenges}
 Due to the limited redundancy that Hamming codes add to the data, they only detect and correct errors when the error rate is low. Another area of challenge for coding theorists is that of developing codes with larger number of errors. BCH and some other designed codes turn out to be inefficient for increasing number of error. Small codes with desired error correction capabilities can be easily developed; developing large codes presents a real practical problem.
\subsection{Future scope}
 Extended Hamming codes achieve a Hamming distance of four, which allows the decoder to distinguish between when at most one-bit error occurs and when two-bit errors occur. Extended Hamming codes are single-error correcting and double-error detecting, abbreviated as \textit{SECDED}. Coding applications have grown rapidly in the past several years and includes the study and discovery of various coding schemes that are used to correct the errors that are introduced during data transmission.\\
\rule{\textwidth}{0.4pt}

\section{Awards and Recognition}
\begin{itemize}
	\item Turing Award, Association for Computing Machinery, 1968
	\item IEEE Emanuel R. Piore Award, 1979
	\item Member of the National Academy of Engineering, 1980
	\item Harold Pender Award, University of Pennsylvania, 1981
	\item IEEE Richard W. Hamming Medal, 1988
	\item Fellow of the Association for Computing Machinery, 1994
	\item Basic Research Award, Eduard Rhein Foundation, 1996
\end{itemize}

\end{document}